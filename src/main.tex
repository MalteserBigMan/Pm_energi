\documentclass[11p]{article}
% Packages
\usepackage{amsmath}
\usepackage{graphicx}
\usepackage[swedish]{babel}
\usepackage[
    backend=biber,
    style=authoryear-ibid,
    sorting=ynt
]{biblatex}
\usepackage[utf8]{inputenc}
\usepackage[T1]{fontenc}
%Källor
\addbibresource{mall.bib}
\graphicspath{ {./images/} }

\title{PMmall \\ \small Fysik 1}
\author{Malte Lindkvist}
\date{\today}

\begin{document}

    \begin{titlepage}
        \begin{center}
            \vspace*{1cm}

            \Huge
            \textbf{Vindkraft}

            \vspace{0.5cm}
            \LARGE
            Vindkraftverk

            \vspace{1.5cm}

            \textbf{Malte Lindkvist}

            \vfill

            Ett PM om energiförsörjning \\
            Fysik 1

            \vspace{0.8cm}

            \includegraphics[width=0.4\textwidth]{NTI Gymnasiet_Symbol_print_svart.png}

            \Large
            Teknikprogrammet\\
            NTI Gymnasiet\\
            Umeå\\
            \today

        \end{center}
    \end{titlepage}
% Om arbetet är långt har det en innehållsförteckning, annars kan den utelämnas

    \newpage
    \section{Inledning}
    El är någonting vi alla använder och kommer använda under de största delar av våra liv, detta betyder att mycket el kommer gå åt, det innebär att vi behöver förnybar energi.
    Det finns flera olika slags förnybar energi, några av dom är vattenkraft, vindkraft och solenergi. Det kallas för en förnybar energikälla eftersom att det ständigt fylls på, t.ex solenergin, solen kommer inte "ta slut" för att vi använder solpaneler medans olja tar slut ifall vi försummar den.

    \section{Inledning}
    Beskriv varför detta ämne är intressant eller viktigt. Vad är syftet med texten?
    \subsection{frågeställningar}
    rada upp dina frågor i punktform
    \begin{enumerate}
        \item Hur fungerar vindkraftverk?
        \item Hur påverkar vindkraftverk miljön?
        \item Hur påverkar vindkraftverk samhället?
    \end{enumerate}

    \section{Resultat}
    Här kommer allt med massor av mer rubriker och underrubriker
    \subsection{Vindkraft, så fungerar det}
    Ett vindkraftverk genererar energi genom vinden, ett vindkraftverk har en rotor (bladen som snurrar) som driver en generator som sedan producerar el vilket överförs till elnätet.

    \includegraphics[width=0.4\textwidth]{Vindkraftverk,_principskiss.svg}


    \subsection{Vindkraft, miljöpåverkan}
    Vindkraft i sig påverkar inte miljön på ett negativt sätt, detta kan dock inte sägas om vindkraftverken. Idag är det vanligt att ett vindkraftverk har en rotordiameter på 80-120m och en höjd på 90-100m.
    Resurserna som behövs för bara ett av dessa

    \section{Slutsatser}
    Här kan du dra slutsatser eller sammanfatta ditt resultat

% Mer saker som du kan ha nytta av.

    \section{Referenser}
    Referenser i text kan skrivas på två sätt: Enligt \textcite{Jens} kan man använde två typer av referenser, inbäddade i texten eller efter ett fakta \parencite{Fraenkel}. Ett till test för att se hur det ser ut \parencite[sid 55]{fermi}.

    \section{Annat som kan vara bra att veta}
    Om du vill ha kodstil och få med alla tecken kan du använda verbatim. då kan du skriva \verb|abcd!"#| utan problem...

    Citat skrivs mellan de konstiga symbolerna \verb|``| och \verb|''| för att de ska se bra ut ``se bra ut!''.
    \subsection{En underrubrik}
    \subsubsection{En underunderrubrik}
    \subsection{Ekvationer}
    Det är lätt att skriva matematik i \LaTeX

    \begin{equation}
        F = G \frac{M m}{r^2}
        \label{grav}
    \end{equation}

    Ekvation (\ref{grav}) känner ni igen...

    \subsection{figurer}
    Bilder placeras enklast på detta sätt. placeringen bestämmer \LaTeX och vi kan bara föreslå (h)är, (t)opp eller (b)otten. Ett utropstecken före tvingar lite mer men inte absolut. I bild  visas en varg
    %\begin{figure}[!h]
    %    \includegraphics[width=0.8\textwidth]
    %    \caption{Acceleration-tid diagram. Källa: Impuls Fysik 1}
    %    \label{varg}
    %\end{figure}
    \printbibliography

\end{document}
